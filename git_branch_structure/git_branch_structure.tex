% Based on a template (http://www.texample.net/media/tikz/examples/TEX/class-diagram.tex) by Remus Mihail Prunescu
\documentclass{minimal}
\usepackage[a4paper,margin=1cm,landscape]{geometry}
\usepackage{tikz}
\usepackage{verbatim}
\usepackage[active,tightpage]{preview}
\PreviewEnvironment{tikzpicture}
\setlength\PreviewBorder{5pt}%
\usetikzlibrary{positioning,shapes,shadows,arrows}
\begin{document}
\tikzstyle{abstract}=[rectangle, draw=black, rounded corners,
        text centered, anchor=north, text=black, text width=3cm]
\tikzstyle{master}=[fill=green!30]
\tikzstyle{dev}=[fill=cyan!30]
\tikzstyle{feature}=[fill=orange!30]

\tikzstyle{comment}=[rectangle, draw=black, rounded corners, fill=green,
        text centered, anchor=north, text=white, text width=3cm]
\tikzstyle{myarrow}=[->, >=open triangle 90, thick]
\tikzstyle{line}=[-, thick]
        
\begin{center}
\begin{tikzpicture}[node distance=0.5cm]
    \node (master) [abstract, rectangle split, rectangle split parts=2, master, text width=5cm]
    {
            \textbf{master}
            \nodepart{second}Source code that is stable
    };
    \node (AuxNode00) [text width=0.1cm, below=of master] {};
    \node (dev) [abstract, rectangle split, rectangle split parts=2, below=of AuxNode00, dev, text width=7cm]
    {
    	\textbf{dev}
    	\nodepart{second}Source code that is implemented but not really stable working with the other modules
    };
\node (AuxNode01) [text width=5cm, below=of dev] {};
\node (communicationdev) [abstract, rectangle split, rectangle split parts=2, left=of AuxNode01, dev, text width=4cm]
{
	\textbf{communication-dev}
	\nodepart{second}$\dots$
};
\node (guidev) [abstract, rectangle split, rectangle split parts=2, below=of AuxNode01, dev, text width=5cm]
{
	\textbf{gui-dev}
	\nodepart{second}Module source code that is implemented but not really stable
};
\node (imageprocessingdev) [abstract, rectangle split, rectangle split parts=2, right=of AuxNode01, dev, text width=4cm]
{
	\textbf{image-processing-dev}
	\nodepart{second}$\dots$
};
\node (AuxNode02) [text width=5cm, below=of guidev] {};
\node (guidevfeature) [abstract, rectangle split, rectangle split parts=2, left=of AuxNode02, feature, text width=4cm]
{
	\textbf{gui-dev-feature1}
	\nodepart{second}A feature currently developed for this module
};
\node (guidevfeature2) [abstract, rectangle split, rectangle split parts=2, right=of AuxNode02, feature, text width=4cm]
{
	\textbf{gui-dev-feature2}
	\nodepart{second}$\dots$
};
\node (AuxNode03) [text width=4cm, below=of guidevfeature] {};
\node (guidevfeaturepart) [abstract, rectangle split, rectangle split parts=2, below=of AuxNode03, feature, text width=4cm]
{
	\textbf{gui-dev-feature1-part1}
	\nodepart{second}A part of the feature currently developed for this module
};
\node (guidevfeaturepart2) [abstract, rectangle split, rectangle split parts=2, right=of AuxNode03, feature, text width=4cm]
{
	\textbf{gui-dev-feature1-part2}
	\nodepart{second}$\dots$
};
    
    \draw[myarrow] (dev.north) -- ++(0,0) -| (master.south);
    
    \draw[myarrow] (guidev.north) -- ++(0,0) -| (dev.south);
    \draw[line] (AuxNode01.south) --++(0,0) -| (communicationdev.east);
    \draw[line] (AuxNode01.south) -- ++(0,0) -| (imageprocessingdev.west);
    
    \draw[myarrow] (guidevfeature.east) -- ++(0.8,0) -| (guidev.south);
    \draw[line] (guidevfeature2.west) -- ++(-0.8,0) -| (guidevfeature.east);
    
    \draw[myarrow] (guidevfeaturepart.north) -- ++(0,0) -| (guidevfeature.south);
    \draw[line] (guidevfeaturepart2.west) -- ++(0,0) -| (guidevfeaturepart.north);
        
        
\end{tikzpicture}
\end{center}
\end{document}