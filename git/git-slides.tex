\documentclass[10pt]{beamer}
% Theme
\usetheme{metropolis}
% Get colors
\usepackage{xcolor}
% Get source code syntax highlighting
\usepackage[cache=false]{minted}
% Get CC doc license
\usepackage[type={CC},modifier={by-nc-sa},version={3.0}]{doclicense}

% Document information
\newcommand*{\mytitleTitle}{Git Basics}
\newcommand*{\mytitleSubtitle}{StuPro Cheat Slides}
\newcommand*{\mytitle}{\mytitleSubtitle: \mytitleTitlet}
\newcommand*{\myauthor}{Niklas Mikeler}
\newcommand*{\mysubject}{StuPro - Multi-Kamera-System-Panorma-Aufnahmen - SS19}
\newcommand*{\mydate}{18.04.2019}

% Custom colors
\definecolor{lightGreyCustom}{HTML}{e8e8e8}
\definecolor{greyCustom}{HTML}{929292}
%\setbeamercolor{normal text}{fg=black,bg=white}
%\setbeamercolor{alerted text}{fg=highlightedText}

% Document methods
\newcommand{\urlCustom}[1]{\href{https://#1}{\textcolor{greyCustom}{\mintinline{sh}{#1}}}}

% Title page
\title{\mytitleTitle}
\subtitle{\mytitleSubtitle}
\date{\mydate}
\subject{\mysubject}
\keywords{\mytitleTitle, \mytitleSubtitle, \mysubject}
\author{\myauthor}
\institute{Universit{\"a}t Stuttgart}
\titlegraphic{\hfill\includegraphics[height=1.5cm]{resources/picture_unistuttgart_logo_deutsch.pdf}}

% Footer
\setbeamertemplate{frame footer}{\mytitleSubtitle: \mytitleTitle \ by \myauthor}

\begin{document}

\maketitle


\begin{frame}{Table of contents}
\setbeamertemplate{section in toc}[sections numbered]
\tableofcontents[hideallsubsections]
\end{frame}


\section{The concept}

\begin{frame}{The concept $\gg$ What is git?}
\begin{itemize}
	\item Git helps you to keep track of the changes/history of the source code in your directory/repository
	\item This is not done by saving for each file a \textit{file-$\Delta$}, but by making a snapshot of all the files every time a change occurs (it even makes a checksum to be sure that nobody can change something unnoticed)
	\item Nearly everything is performed locally without a network connection - like when you want to look at the complete history of the project you can just do it
	\item When you fine with what you've done you can then upload all the changes at once to a server
\end{itemize}
\end{frame}

\begin{frame}{The concept $\gg$ The 3 sections and 3 stages of files}
\begin{itemize}
	\item The 3 file stages:
	\begin{itemize}
		\item Committed: \textit{The data is safely stored in the local database}
		\item Modified: \textit{The data is modified but not committed}
		\item Staged: \textit{The modified data is marked to go into the next commit snapshot}
	\end{itemize}
	\item The 3 main sections of any git project:
	\begin{itemize}
		\item \mintinline{sh}{.git} directory: \textit{In here all the metadata and the snapshot database is stored (This gets copied when you clone a git repository)}
		\item Working directory: \textit{A single checkout of one version of the project pulled from the database from the \mintinline{sh}{.git} directory}
		\item Staging area: \textit{One file that stores information about what will go into the next commit}
	\end{itemize}
\end{itemize}
\end{frame}

\begin{frame}{The concept $\gg$ The basic git workflow I}
\begin{enumerate}
	\item You modify files in the project directory ($=$ working tree) \\
	\inputminted[bgcolor=lightGreyCustom,fontsize=\scriptsize]{sh}{./resources/git_workflow_01_modify.sh}
	\item You select which changes should go into the next commit snapshot by adding these changes to the staging area \\
	\inputminted[bgcolor=lightGreyCustom,fontsize=\scriptsize]{sh}{./resources/git_workflow_02_stage.sh}
	\item You commit all the staged changes with message/description permanently to the \mintinline{sh}{.git} directory \\
	\inputminted[bgcolor=lightGreyCustom,fontsize=\scriptsize]{sh}{./resources/git_workflow_03_commit.sh}
\end{enumerate}
\end{frame}

\begin{frame}{The concept $\gg$ The basic git workflow II}
\begin{enumerate}\setcounter{enumi}{3}
	\item You can too remove files from the working tree (unstage) \\
	\inputminted[bgcolor=lightGreyCustom,fontsize=\scriptsize]{sh}{./resources/git_workflow_04_unstage.sh}
	\item You can remove all files from the working tree to reset it\\
	\inputminted[bgcolor=lightGreyCustom,fontsize=\scriptsize]{sh}{./resources/git_workflow_05_reset.sh}
	\item You can get an report of which files are added and which are untracked (modified but not staged) \
	\inputminted[bgcolor=lightGreyCustom,fontsize=\scriptsize]{sh}{./resources/git_workflow_06_status.sh}
\end{enumerate}
\end{frame}

\section{Remote repository}

\begin{frame}{Remote repository $\gg$ The basics I}
\begin{itemize}
\item You can also save the git database on a server
\item Through this a team of people can access and change this database together
\end{itemize}
\begin{enumerate}
	\item You can download ($=$ clone) this remote repository (its database) \\
	\inputminted[bgcolor=lightGreyCustom,fontsize=\scriptsize]{sh}{./resources/git_remote_repository_01_clone.sh}
\end{enumerate}
\end{frame}

\begin{frame}{Remote repository $\gg$ The basics II}
\begin{enumerate}\setcounter{enumi}{1}
	\item You can always download the latest state of the database ($=$ all the commits from all the other people) \\
	\inputminted[bgcolor=lightGreyCustom,fontsize=\scriptsize]{sh}{./resources/git_remote_repository_02_pull.sh}
	\item And of course you can upload your committed changes \\
	\inputminted[bgcolor=lightGreyCustom,fontsize=\scriptsize]{sh}{./resources/git_remote_repository_02_push.sh}
\end{enumerate}
\begin{itemize}
	\item The problem is that in such a situation many things can suddenly happen like:
	\begin{itemize}
		\item Someone was editing the same file as you currently do and pushes his changes before you to the server
	\end{itemize}
\end{itemize}
\end{frame}

\section{Merging/Conflicts}

\begin{frame}{Merging $\gg$ Why is it making my life so much worse?}
\begin{itemize}
	\item Without merging a situation where two people edit the same file and both push their changes to the server would be really painful
	\item The git protocol in this situation first checks if the changes are in conflict (if both user added lines in different places it just merges them)
	\item But if both user edited on the same line or something else git will ask you to solve this mess
\end{itemize}
\end{frame}

\begin{frame}{Merging $\gg$ Create a merge conflict}
\begin{itemize}
	\item The following script simulates the creation of a merge conflict in a file on one line:
	\href{https://jonathanmh.com/how-to-create-a-git-merge-conflict/}{\beamergotobutton{Link}} 
	\inputminted[bgcolor=lightGreyCustom,fontsize=\scriptsize]{sh}{./resources/git_merging_03_create_conflict.sh}
\end{itemize}
\end{frame}

\begin{frame}{Merging $\gg$ How does a merge conflict look?}
\begin{itemize}
	\item It looks complicated but is easy to solve if you know why it looks that way:
	\inputminted[bgcolor=lightGreyCustom,fontsize=\scriptsize]{sh}{./resources/git_merging_04_create_conflict_output.sh}
	\begin{itemize}
		\item The area above the \mintinline{diff}{=======} is the content on this line you currently have, and the other is the content that another person made before you on the same line
		\item As simple as it comes you are now the person to solve this by just deciding which of the lines you want - or a combination of both?
		\item If you have decided - it's just a text file... edit it so that the file looks for example like the following:
		\inputminted[bgcolor=lightGreyCustom,fontsize=\scriptsize]{sh}{./resources/git_merging_05_create_conflict_solved.sh}
	\end{itemize}
\end{itemize}
\end{frame}

\begin{frame}{Merging $\gg$ Vim $\gg$ How to enter/exit/save}
\begin{itemize}
\item It's quite possible if you use git in the terminal that suddenly a weird text editor pops up
\item This text editor is called vim and it's no that simple to use it
\end{itemize}
\begin{enumerate}
	\item Vim has 2 modes/states: Insert and Command (default)
	\item In short: in Command you can enter commands like \mintinline{sh}{:w} (save the file), \mintinline{sh}{:wq} (write and quit), \mintinline{sh}{:q} (quit) or \mintinline{sh}{:fq} (force quit)
	\item To get from this mode to the insert mode where you can edit text just like in a normal text editor simply press the key \mintinline{sh}{i}
	\item To get back from there to the Command mode press the \mintinline{sh}{Esc} key
\end{enumerate}
\end{frame}

\section{Branches}

\begin{frame}{Branches $\gg$ Create, delete}
\begin{center}
\begin{itemize}
\item There is still the problem that if you have many people in one project who all constantly push to one repository that the history is unusable/unreadable $+$ constantly merge conflicts for everyone
\item This problem is solved by having branches that can be based on any commit snapshot (initially everything is on the \mintinline{sh}{master} branch)
\end{itemize}
\begin{enumerate}
	\item You can ask which branch you are on \\
	\inputminted[bgcolor=lightGreyCustom,fontsize=\scriptsize]{sh}{./resources/git_branches_01_branch.sh}
	\item You can create a new branch based on the currently latest commit \\
	\inputminted[bgcolor=lightGreyCustom,fontsize=\scriptsize]{sh}{./resources/git_branches_02_create.sh}
	\item And of course you can also switch your working tree to any branch \\
	\inputminted[bgcolor=lightGreyCustom,fontsize=\scriptsize]{sh}{./resources/git_branches_03_checkout.sh}
\end{enumerate}
\end{center}
\end{frame}

\begin{frame}{Branches $\gg$ Merge}
\begin{center}
	\begin{enumerate}\setcounter{enumi}{3}
		\item If it's the right time and your feature was implemented on the branch you can merge it with another branch \\
		\inputminted[bgcolor=lightGreyCustom,fontsize=\scriptsize]{sh}{./resources/git_branches_04_merge.sh}
	\end{enumerate}
\end{center}
\end{frame}

\section{Rebase}

\begin{frame}{Rebase $\gg$ What can it do and when should you use it?}
\begin{center}
\begin{itemize}
\item Rebase makes it possible to
\begin{enumerate}
	\item Combine commits
	\item Forget commits
	\item Rename commits
\end{enumerate}
\item Git will internally create new commits which invalidates the checksums, so only use this if you know what it does!
\item \textbf{Don't rebase public history!!} Only do it \textbf{\textit{before}} you push it to the server!
\end{itemize}
\end{center}
\end{frame}

\begin{frame}{Rebase $\gg$ Combine/delete commits from history}
\begin{center}
	\begin{enumerate}
		\item First you need to find the checksum of the origin commit since when you want to change it
		\begin{itemize}
			\item For example if you want to merge the last 4 commits you want to find the 4th latest checksum
		\end{itemize}
	\inputminted[bgcolor=lightGreyCustom,fontsize=\scriptsize]{sh}{./resources/git_rebase_01_checksums.sh}
	\item Then start the rebase
	\inputminted[bgcolor=lightGreyCustom,fontsize=\scriptsize]{sh}{./resources/git_rebase_02_rebase.sh}
	\item Then you get a document where you can \mintinline{sh}{p} pick (keep), \mintinline{sh}{r} reword (rename title + description), \mintinline{sh}{d} drop (remove), \mintinline{sh}{s} squash commits (keep it, but integrate into the previous commit), \mintinline{sh}{f} fix-up commits (keep it, but integrate into the previous commit and forget commit message)
	\item If you save the file all your changes will be applied
	\end{enumerate}
\end{center}
\end{frame}

\begin{frame}{Rebase $\gg$ Example file}
\begin{center}
	\begin{enumerate}
		\item Show the rebase text file
	\end{enumerate}
\end{center}
\end{frame}

\section{Undos}

\begin{frame}{Undos $\gg$ Staged files}
\inputminted[bgcolor=lightGreyCustom,fontsize=\scriptsize]{sh}{./resources/git_undos_01_stage_files.sh}
\end{frame}

\begin{frame}{Undos $\gg$ Last commit}
\inputminted[bgcolor=lightGreyCustom,fontsize=\scriptsize]{sh}{./resources/git_undos_02_last_commit.sh}
\end{frame}

\begin{frame}{Undos $\gg$ Hard reset all changes to a commit}
\inputminted[bgcolor=lightGreyCustom,fontsize=\scriptsize]{sh}{./resources/git_undos_03_hard_reset.sh}
\end{frame}


\section{Other}

\begin{frame}{Other $\gg$ Don't track files using a \mintinline{diff}{.gitignore} file}
\begin{itemize}
	\item If your for example build binaries in the project directory git will always track it
	\item This is not a problem, but something that you can get behind by telling git which files should it shouldn't track (like \mintinline{diff}{build.exe}), because only the source code should end up in the repository
	\item Therefore just create a text file called \mintinline{diff}{.gitignore} in which you can add rules for which files to ignore in this and all sub directories
	\inputminted[bgcolor=lightGreyCustom,fontsize=\scriptsize]{sh}{./resources/.gitignore}
\end{itemize}
\end{frame}

\begin{frame}{Other $\gg$ Git submodules}
\end{frame}

\begin{frame}{Other $\gg$ Sign commits}
\end{frame}

\begin{frame}{Other $\gg$ Git subroutines commits}
\end{frame}


\begin{frame}{Links}

\begin{itemize}
\item\urlCustom{https://git-scm.com/book/en/v2}
\item\urlCustom{http://rogerdudler.github.io/git-guide/2}
\item\urlCustom{https://www.git-tower.com/learn/git/faq/}
\end{itemize}

\begin{center}\doclicenseThis\end{center}

\end{frame}

\end{document}
